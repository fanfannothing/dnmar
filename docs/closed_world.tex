\documentclass[12pt]{article}

\newcommand{\E}[1]{{\mathbf{E}[#1]}}
\newcommand{\Prob}[1]{{\mathbf{Pr}[#1]}}
\newcommand{\ProbSub}[2]{{\mathbf{Pr}_{#1}[#2]}}
\newcommand{\Var}[1]{{\mathbf{Var}[#1]}}
\newcommand{\Cov}[1]{{\mathbf{Cov}[#1]}}

\usepackage[authoryear,round]{natbib}
\usepackage{amsmath}
\usepackage{graphicx}
\usepackage{comment}
\usepackage{url}

\usepackage{subfigure}

\begin{document}

\date{}
\title{Modeling Distant Supervision and Heuristically Generated Training Data with Latent Variables}
\author{}
\maketitle

\bibliographystyle{plainnat}

\section{Challenge}
\begin{description}
  \item[{\bf Incomplete Databases}]:
    \begin{itemize}
    \item Information is often missing from databases (e.g. the list of cities in Freebase might be incomplete).
    \item This leads to misleading training data, for example supposing \emph{Amazon} is listed as a {\sl COMPANY},
      but not a {\sl LOCATION}, the sentence \emph{I am flying to the Amazon} would be mis-labeled as referring
      to a {\sl company}.
    \end{itemize}
  \item[{\bf Missing Entries} (\emph{Semi-Supervised Distant Supervision)]:
    \begin{itemize}
      \item Many items (e.g. entities, or entity-pairs) will not appear at all in the database
      \item If we are sufficently confident, extractions involving these entities should be treated as if they were added to the database
        and used for futher Distant Supervision.
        %TODO: Example
    \end{itemize}
\end{description}

\section{Solution: Relax the Closed World Assumption}
To address both of these challenges, I believe we should investigate an approach based a generative model of both the text and the database which
preserves ambiguity over which facts are correct/incorrect rather than assuming all facts in the database are true, and all facts not in the database
are false (closed world assumption).

\subsection{Background: Current Approach}


\subsection{Proposed Approach}

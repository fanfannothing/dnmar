\documentclass[12pt]{article}

\newcommand{\E}[1]{{\mathbf{E}[#1]}}
\newcommand{\Prob}[1]{{\mathbf{Pr}[#1]}}
\newcommand{\ProbSub}[2]{{\mathbf{Pr}_{#1}[#2]}}
\newcommand{\Var}[1]{{\mathbf{Var}[#1]}}
\newcommand{\Cov}[1]{{\mathbf{Cov}[#1]}}

\usepackage[authoryear,round]{natbib}
\usepackage{amsmath}
\usepackage{graphicx}
\usepackage{comment}
\usepackage{url}

\usepackage{subfigure}

\begin{document}

\date{}
\title{Relaxing The Closed World Assumption in Distant Supervision with a Generative Model of Corpora and Databases}
\author{}
\maketitle

\bibliographystyle{plainnat}

\section{Challenge}
\begin{description}
  \item[{\bf Incomplete Databases}]:
    \begin{itemize}
    \item Facts are often omitted from databases.
    \item This can result in misleading training data, for example supposing \emph{Amazon} is listed as a {\sl COMPANY},
      but not a {\sl LOCATION}, the sentence \emph{I am flying to the Amazon} would be mis-labeled as referring
      to a {\sl COMPANY}.
    \end{itemize}
  \item[{\bf Missing Entries} (\emph{Semi-Supervised Distant Supervision})]:
    \begin{itemize}
      \item Many items (e.g. entities, or entity-pairs) will not appear at all in the database
      \item If we are sufficently confident, extractions involving these entities should be treated as if they were added to the database
        and used for futher Distant Supervision.
        %TODO: Example
    \end{itemize}
\end{description}

\section{Solution}
To address both of these challenges, I believe we should investigate an approach based a generative model of both the text and the database which
preserves ambiguity over which facts are correct/incorrect rather than assuming all facts in the database are true, and all facts not in the database
are false (closed world assumption).

\subsection{Background: Previous Approach}
In previous work \cite{Ritter11}, we proposed an approach to distant supervision based on a Bayesian Network (Latent Dirichlet Allocation) representation of our data
in connection with a set of constraints on it's parameters.  More specifically:
\begin{itemize}
  \item We investigated weakly supervised named entity classification (e.g. input=extracted named entities and contexts + database, output=model and classification of named entities)
  \item Entities are modeled as mixtures of types (analogy from topic modeling: entities $\rightarrow$ documents, contexts $\rightarrow$ words, types $\rightarrow$ words.
  \item The set of possible types for each entity are constrained based on it's possible types according to a set of mutually exclusive dictionaries gathered from Freebase.
\end{itemize}

\subsection{Proposed Approach}

\bibliography{bib}
\end{document}
